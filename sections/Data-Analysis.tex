\section{Data Analysis}

\subsection{Exploratory Data Analysis}

\subsubsection{Dataset Overview}

Data Description

The Forest CoverType dataset from the UCI Machine learning repository is used in this study to model the forest cover classification as a supervised multiclass problem. The initial exploration reveal that:
\begin{itemize}
    \item \textbf{Total samples}: 581,012
    \item \textbf{Total features}: 54
    \item \textbf{Missing values}: 0 (no imputation required)
    \item \textbf{Data types}: 10 float64 (continuous), 44 int64 (binary)
    \item \textbf{Memory usage}: $\sim$248 MB
\end{itemize}

\subsubsection{Feature Type Analysis}
The Feature set is composed into two main groups. The first group includes ten continuous variables describing terrain characteristics and spatial distances, these values meansured on hetorogenous scales, which whill be consider more carefully with scale-sensitive machine learning models. The second groups consists of forty-four binary indicator variables. All of these values are encoded using one-hot represenation with values in {0,1}. As these features are already normalized into binary indicators, therefor no additional scaling is required. These features include:

\begin{enumerate}
    \item \textbf{Continuous Features (n=10)}: 
    \begin{itemize}
        \item Topographic: Elevation, Aspect, Slope
        \item Distance metrics: to hydrology, roadways, fire points
        \item Solar radiation: Hillshade at 9am, noon, 3pm
    \end{itemize}
    
    \item \textbf{Binary Features (n=44)}:
    \begin{itemize}
        \item Wilderness Areas (n=4): One-hot encoded wilderness designation
        \item Soil Types (n=40): One-hot encoded soil classification
    \end{itemize}
\end{enumerate}

\subsection{Statistics Analysis}

\subsubsection{Descriptive Statistics}

Table \ref{tab:descriptive_stats} presents descriptive statistics for the ten continuous variables. The results indicate subtsantial hetorogeneity in scale and dispersion across all features. For instance, \texttt{Elevation} has a mean of 2959.4 and a relatively small conefficient of variation ($9.5\%$). In the opposite, \texttt{Vertical\_Distance\_To\_Hydrology} have a coefficient of variation, reflecting a wide range around the mean of the feature values. Distance-related variabes generally show large standard deviations, showing that geographic features are strongly affect the convertype

Regarding distributional shape,  most features in the dataset show decent skewness (i.e. $|$skewness$|$ < 1), showing a approximate symmetry in the dataset.

Overall, the dataset continuous features have a wide range of scales and variability across features, therefore proff the effectiveness for classification task, prior for machine learing algorithm that sensitive to feature magnitude. 

\begin{table}[h]
\centering
\caption{Descriptive Statistics for Continuous Features}
\label{tab:descriptive_stats}
\small
\begin{tabular}{@{}lrrrrrrr@{}}
\toprule
\textbf{Feature} & \textbf{Mean} & \textbf{Std} & \textbf{Min} & \textbf{Max} & \textbf{Skew} & \textbf{Kurt} & \textbf{CV\%} \\ \midrule
Elevation & 2959.4 & 279.8 & 1859 & 3858 & $-0.08$ & $-0.41$ & 9.5 \\
Aspect & 155.7 & 111.9 & 0 & 360 & 0.32 & $-1.26$ & 71.9 \\
Slope & 14.1 & 7.5 & 0 & 66 & 0.91 & 1.04 & 53.2 \\
H\_Dist\_Hydrology & 269.4 & 212.5 & 0 & 1397 & 1.09 & 1.32 & 78.9 \\
V\_Dist\_Hydrology & 46.4 & 58.3 & $-173$ & 601 & 1.78 & 5.82 & 125.6 \\
H\_Dist\_Roadways & 2350.1 & 1559.3 & 0 & 7117 & 0.76 & 0.25 & 66.4 \\
Hillshade\_9am & 212.1 & 26.8 & 0 & 254 & $-1.05$ & 1.81 & 12.6 \\
Hillshade\_Noon & 223.3 & 19.8 & 0 & 254 & $-1.84$ & 5.95 & 8.9 \\
Hillshade\_3pm & 142.5 & 38.3 & 0 & 254 & 0.23 & $-0.54$ & 26.9 \\
H\_Dist\_Fire\_Points & 1511.1 & 1099.9 & 0 & 7173 & 0.96 & 0.87 & 72.8 \\ \bottomrule
\end{tabular}
\end{table}

